\documentclass[12pt]{scrartcl}

\usepackage[T1]{fontenc}
\usepackage[utf8]{inputenc}
\usepackage{amsmath}
\usepackage{amsfonts}
\usepackage{amssymb}
\usepackage{amsbsy}
\usepackage{graphicx}
\usepackage{float}
\usepackage{array}
\usepackage{booktabs}
\usepackage{multirow}
\usepackage{bm}
\usepackage{verbatim}
\usepackage[english]{babel}
\usepackage{color}
\usepackage{url}
\usepackage{fancybox}
\usepackage[stable]{footmisc}
\usepackage[format=plain,labelfont=bf]{caption}
\usepackage{fancyhdr}
\usepackage{natbib}
\usepackage{calc}
\usepackage{textcomp}
\usepackage[pdftex,pdfborder={0 0 0}]{hyperref}
\usepackage{pdfpages}
\usepackage{tikz}
\usetikzlibrary{shapes,arrows,decorations.pathmorphing}

\renewcommand*{\familydefault}{\sfdefault}

% Margins
\addtolength{\textheight}{1.0cm}
\addtolength{\oddsidemargin}{-0.5cm}
\addtolength{\evensidemargin}{-0.5cm}
\addtolength{\textwidth}{1.0cm}
\parindent=0em

% New math commands
\newcommand{\overbar}[1]{\mkern 1.5mu\overline{\mkern-1.5mu#1\mkern-1.5mu}\mkern 1.5mu}
\DeclareMathOperator{\diag}{diag}
\DeclareMathOperator{\Diag}{Diag}
\DeclareMathOperator{\tr}{tr}
\DeclareMathOperator{\Cov}{Cov}
\DeclareMathOperator{\Cor}{Cor}
\newcommand\independent{\protect\mathpalette{\protect\independenT}{\perp}}
\def\independenT#1#2{\mathrel{\setbox0\hbox{$#1#2$}%
\copy0\kern-\wd0\mkern4mu\box0}}

% Appropriate font for \mathcal{}
\DeclareSymbolFont{cmmathcal}{OMS}{cmsy}{m}{n}
\DeclareSymbolFontAlphabet{\mathcal}{cmmathcal}

% Set subscript and superscripts positions
\everymath{
\fontdimen13\textfont2=5pt
\fontdimen14\textfont2=5pt
\fontdimen15\textfont2=5pt
\fontdimen16\textfont2=5pt
\fontdimen17\textfont2=5pt
}

% Bibliography style
\setlength{\bibsep}{1pt}

% Part
\renewcommand\partheadstartvskip{\clearpage\null\vfil}
\renewcommand\partheadmidvskip{\par\nobreak\vskip 20pt\thispagestyle{empty}}
\renewcommand\partheadendvskip{\vfil\clearpage}
\renewcommand\raggedpart{\centering}

\begin{document}
\title{Diffusion and Matern functions}
\author{Benjamin Ménétrier}
\date{Last update: \today}

\thispagestyle{empty}

\maketitle
\begin{center}
Documentation for the code "BUMP", distributed under the CeCILL-C license.\\
Copyright \copyright 2015-... UCAR, CERFACS, METEO-FRANCE and IRIT
\end{center}

\section{Goal}
The aim is to find a practical way to compute the Matern function:
\begin{align}
c_\nu(\tilde{r}) = \alpha_\nu \tilde{r}^\nu K_\nu(\tilde{r})
\end{align}
where $K_\nu$ is the modified Bessel function of second kind and the factor $\alpha_\nu$ ensures that $c_\nu$ is normalized ($c_\nu(0) = 1$).

\section{Explicit formula}
For the particular case where $\nu-1/2 \in \mathbb{Z}$, there is a special formula for the modified Bessel function of second kind (\url{http://functions.wolfram.com/Bessel-TypeFunctions/BesselK/introductions/Bessels/05/}):
\begin{align}
K_\nu(\tilde{r}) = \sqrt{\pi} e^{-\tilde{r}} \sum_{j=0}^{\lfloor |\nu|-1/2 \rfloor} \frac{\left(|\nu|-1/2+j\right)!}{j!\left(|\nu|-1/2-j\right)!}(2\tilde{r})^{-j-1/2}
\end{align}
Thus for $\nu>0$:
\begin{align}
c_\nu(\tilde{r}) & = \alpha_\nu \tilde{r}^\nu K_\nu(\tilde{r}) \\
& = \alpha_\nu \frac{\sqrt{\pi}}{2^\nu} e^{-\tilde{r}} \sum_{j=0}^{\nu-1/2} \frac{\left(\nu-1/2+j\right)!}{j!\left(\nu-1/2-j\right)!} (2\tilde{r})^{\nu-1/2-j} \\
\end{align}

\section{Practical computation}
For the implicit diffusion, $\nu = M-d/2$ where $M \in \mathbb{N}$ is the number of implicit steps, and $d$ the dimension. So only the cases $d=1$ and $d=3$ are valid here:
\begin{align}
c_\nu(\tilde{r}) & = \alpha_\nu \frac{\sqrt{\pi}}{2^{M-d/2}} e^{-\tilde{r}} \sum_{j=0}^{M-(d+1)/2} \beta_{j,M-(d+1)/2} (2\tilde{r})^{M-(d+1)/2-j}
\end{align}
where
\begin{align}
\beta_{j,M} & = \frac{\left(M+j\right)!}{j!\left(M-j\right)!}
\end{align}
can be computed recursively:
\begin{align}
\beta_{0,M} & = 1
\end{align}
and
\begin{align}
\beta_{j+1,M} & = \frac{\left(M+j+1\right)!}{(j+1)!\left(M-(j+1)\right)!} \\
& = \frac{(M+j+1)(M-j)}{j+1} \beta_{j,M}
\end{align}
Thus, the normalization becomes:
\begin{align}
c_\nu(0) & = \alpha_\nu \frac{\sqrt{\pi}}{2^{M-d/2}} \beta_{M-(d+1)/2,M-(d+1)/2} = 1
\end{align}
leading to
\begin{align}
\alpha_\nu = \frac{2^{M-d/2}}{\sqrt{\pi}} \frac{1}{\beta_{M-(d+1)/2,M-(d+1)/2}}
\end{align}
And finally:
\begin{align}
c_\nu(\tilde{r}) = \frac{1}{\beta_{M-(d+1)/2,M-(d+1)/2}} e^{-\tilde{r}} \sum_{j=0}^{M-(d+1)/2} \beta_{j,M-(d+1)/2} (2\tilde{r})^{M-(d+1)/2-j}
\end{align}
The case $d=2$ would require another method.

\bibliographystyle{mybib-en}
\bibliography{diffusion_matern_function}
\end{document}